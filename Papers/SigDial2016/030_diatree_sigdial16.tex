\documentclass[11pt]{article}
\usepackage{acl2016}
\usepackage{times}
% \usepackage[round]{natbib}
\usepackage{latexsym}
\usepackage[utf8]{inputenc}
\usepackage[font=small,labelfont=bf]{caption}
\usepackage{amsmath}
\usepackage{multirow}
\usepackage{appendix}
\usepackage{url}
\usepackage{tikz}
% \usepackage{avm}
% \avmfont{\sc}
% \avmoptions{sorted}
% \avmvalfont{\rm}
% \avmsortfont{\scriptsize\it}
\usepackage{remreset}
\usepackage{pifont}
% \newcommand{\cmark}{\ding{53}}%
\usepackage{graphicx}
\usepackage{wrapfig}
\usepackage{verbatim} 
\usepackage{linguex}
\usepackage{qtree}
%\usepackage{algorithm}% http://ctan.org/pkg/algorithms
%\usepackage{algpseudocode}% http://ctan.org/pkg/algorithmicx
\usepackage{amsmath,amsfonts,amssymb}

\newcommand{\argmax}{\operatornamewithlimits{argmax}}


\newcommand{\das}[1]{\emph{//das: #1//}}
\newcommand{\crk}[1]{\emph{//crk: #1//}}
\newcommand{\todo}[1]{\textcolor{green}{\emph{//todo: #1//}}}


\newcommand{\sds}[0]{\textsc{sds}}
\newcommand{\nlu}[0]{\textsc{nlu}}
\newcommand{\rmrs}[0]{\textsc{rmrs}}
\newcommand{\ep}[0]{\texttt{EP}}
\newcommand{\inprotk}[0]{\textsc{InproTK}}
\newcommand{\sium}[0]{\textsc{sium}}
\newcommand{\asr}[0]{\textsc{asr}}
\newcommand{\dm}[0]{\textsc{dm}}
\newcommand{\ui}[0]{\textsc{ui}}
\newcommand{\iu}[0]{\textsc{iu}}
\newcommand{\rr}[0]{\textsc{rr}}
\newcommand{\conf}[0]{\textsc{conf}}
\newcommand{\pa}[0]{\textsc{pa}}

%Document Head
% \dochead{CLV2 Class File Manual}

\title{An Incremental System that Graphically Signals Speech Understanding}

\author{Casey Kennington \and David Schlangen}
%\affil{CITEC, Bielefeld University}

% \author{David Schlangen}
% \affil{Bielefeld University}

\begin{document}
% \affil{Publishing / SPi}

\maketitle

% abstract should be 150-250 words
\begin{abstract}

\end{abstract}

\section{Introduction}
\label{section:intro}

Suppose a native speaker of English is listening to and transcribing the speech of a native Spanish speaker with one important consideration: the native English speaker doesn't understand Spanish. Upon inspection, it could very well be the case that the transcription is somewhat accurate; the words are segmented properly and the spelling seems to mostly be correct. This illustrates the disconnect between speech reccognition (\asr) and natural language understanding (\nlu); i.e., \emph{intent recognition}: simply being able to recognise words is only the beginning of understanding speech. 

Signaling understanding is an important aspect of spoken dialogue between humans. Humas signal understanding by providing feedback such as nodding, spoken backchannels (e.g., \emph{uh-huh}) or by performing some kind of action \cite{Clark1996}. Crucially, this feedback is provided as the utterance unfolds, such that installments of speech given by the speaker are confirmed as being understood before the speaker commits to continuing with speaking. Current speech-based personal assistants (\pa s) don't work like this, however; rather, they require spoken input from users (i.e., speakers) as complete intents allowing the \pa\ to signal understanding only by displaying the \asr\ (which, as illustrated above, does not denote \emph{understandig}) and carrying out the requested task--the user can only determine if her intent was recognised if the correct task was carried out. Moreover, just how is a user to know what kind of things a \pa\ could potentially understand; i.e., what are the \emph{affordances}?\footnote{For example, current \pa s run a query in a search engine if a request is not understood.}

Current virtual personal assistants require their users to either formulate complex intents in one utterance (as in "make a phone call to Peter Miller on his mobile phone") or go through tedious sub-dialogues ("phone call -- who would you like to call? -- Peter Miller -- I have a mobile number and a work number. Which one do you want?"). This is not how one would interact with a human assistant, where the request would be naturally structured into smaller chunks that individually get acknowledged ("can you make a connection for me? - sure - with Peter Miller - uh huh - on his mobile - got it"). We present a mixed voice/graphical interface that also follows this principle of "early closure on minimal information units".

To address this, some \pa s attempt to learn a \emph{user model} so as to predict what it is the user wants from the system requiring little or no spoken input from the user. These can range on a continuum of \emph{non-predictive} systems that don't attempt to learn anything about the user, to \emph{fully-predictive} systems (such as Google Now) that attempt to predict what a user will utter before the user even utters it (e.g., a user receives a notification on her device with the text: ``Do you want directions to a French restaurant in the downtown area?''). 

In this paper, we present a \pa\ that can provide feedback of understanding to the user \emph{incrmentally} as the user's utterance unfolds. Indeed, our system allows users to make requests in installments instead of fully thought-out requests. Our system does this by displaying ongoing understanding at each recognised word to the user in an intuitive tree-like graphical interface that can be displayed on a mobile devide. We evaluate our system by directing participants to perform tasks using it under non-incremental (i.e., traditional \asr\ endpointing) and incremental conditions and then asking the participants to compare the two conditions with questionaires. We further compared a non-predictive version with a version that attempted to automatically predict the intent of the participant based on previous experiences with the user. We report that the participants found the interface intuitive and easy to understand, that they recognised that at times the system responded as they were speaking, and that at times the system could predict what task they waneted the system to do with little input from the user. 

The remainder of this paper is organised as follows: in the following section, we provide a discussion of related work, then describe our system in Section~\ref{section:system_def}. We then explain the experiments and give the results in Section~\ref{section:experiments}. We then give a final discussion, ideas for future work, and conclude. 


\section{Related Work}
\label{section:related_work}

This work brings together and builds upon several threads of other previous research. \cite{Chai2014} attempted to address misalignments in common ground \cite{clarkschaefer:contrdis} between systems (in their case, robots) and humans by informing the human of the internal state of the system. We take this idea and apply it a \pa\ by displaying the internal state of the system to the user in an intuitive, tree-like structure (explained in Section~\ref{section:display}), allowing the user to determine if understanding has taken place by the system. Such information presentation is a way of providing feedback and backchannels to the user. \cite{Dethlefs2015} provides a good review of work that shows that backchannels facilitate grounding, feedback, and clarifications in human spoken dialogue. The apply an \emph{information density} approach to determining when to backchannel using speech. Because we don't backchannel using speech here, there is no potential overlap between the human user and the system; rather, our system can display backchannels and ask clarifications without interrupting (that is, frustrating) the user. 

Though different in many ways, our work is similar in some regards to \cite{Larsson2011}, which displays information to the user and allows the user to navigate the display itself (e.g., by saying \emph{up} or \emph{down} in a menu list), functionality that we intent to apply to our display in future work. 

Some of the work here is inspired by the \emph{Microsoft Language Understanding Intelligent Service} (LUIS) project \cite{Williams2015_sigdial}. While our system by no means achieves the scale that LUIS does, we offer here an open source LUIS-like system (with the important addition of the display interface) that is authorable (using JSON files; we leave authoring using a web interface--like LUIS--to future work), extendible (affordances can be easily added), incremental (going beyond LUIS), trainable (i.e,. can learn from examples, but can still function well without examples), and can learn through interacting (left for future work, but here we apply a simplified user model that learns during interaction). 

\section{System Description}
\label{section:system_def}

This section introduces and describes our \sds, which is modularised into four main components: automatic speech recognition (\asr), natural langauge understanding (\nlu), dialogue management (\dm), and the user interface (\ui) which, as explained below, is visualised as a right-branching tree. For the remainder of this section, each module is exlpained in turn. First, however, we explain what is meant by incremental processing and the role that plays in the work presented here. 

\begin{figure}[ht]
  \centering
      \includegraphics[width=0.5\textwidth]{figures/sig16-overview.pdf}	
      \caption{Overview of system made up of \asr\ which takes in a speech signal and produces transcribed words, \nlu, which takes words and produces a slots in a frame, \dm\ which takes slots and produces a decision for each, and the \ui\ which displays the state of the system. \label{fig:overview}}
\end{figure}


\subsection{Incremental Dialogue}

Of prime importance in our \sds--an aspect of our \sds\ that sets it apart from others--is the requirement that it processes \emph{incrementally}. An often-cited concern with incremental processing is regarding informativeness: why act so soon when waiting (even just for a moment) would allow additional information, resutling in more-informed decisions? The trade-off here is all-important: \emph{naturalness} as perceived by the end user who is interacting with the \sds. Indeed, it has been shown that humans perceive incremental systems as being more natural than traditional, turn-based systems \cite{Aist2006,Skantze2009,skantze2010sigdial,Asri2014}, offer a more human-like experience for the human users \cite{Edlund2008b} and are more satisfying to interact with than non-incremental systems \cite{Aistetal:incrunder-short}. Psycholinguistic research has also shown that humans process (i.e., comprehend) utterances as they unfold and do not wait until the end of an utterance to begin the comprehension process \cite{Tanenhaus1995,Spivey_2002tw}. 

The trade-off between informativeness and naturalness can be reconciled when mechanisms are in place where earlier desicions can be repaired. Such mechanisms were introduced in the incremental unit (\iu) framework for \sds\ \cite{Schlangen2009,Schlangen2011}. Following \cite{kennington-kousidis-schlangen:2014:W14-43}, \sds s based on the \iu-network approach consist of a network of processing \emph{modules}. A typical module takes input from its \emph{left buffer}, performs some kind of processing on that data, and places the processed result onto its \emph{right buffer}. The data are packaged as the payload of \emph{incremental units} (\textsc{iu}s) which are passed between modules. The \textsc{iu}s themselves are also interconnected via  \emph{same level links} (\textsc{sll}) and \emph{grounded-in links} (\textsc{grin}), the former allowing the linking of \textsc{iu}s as a growing sequence, the latter allowing that sequence to convey what \textsc{iu}s directly affect it.  A complication particular to incremental processing is that modules can ``change their mind'' about what the best hypothesis is, in light of later information, thus \textsc{iu}s can be \emph{added}, \emph{revoked}, or \emph{committed} to a network of \textsc{iu}s.

The modules exlpained in the remainer of this section are implemented as \iu-modules and process incrementally. Each will now be explained. 

\subsection{Speech Recognition}

Incremental processing begins with modules that take in input; in the case of our \sds, that is the \asr\ component. Incremental \asr\ must transcribe uttered speech into words and words must be forthcoming from the \asr\ as early as possible (i.e., the \asr\ must not wait for endponiting in order to act). Each module that follows must also process incrementally, acting in lock-step upon input as it is received. Incremental \asr\ is not new \cite{baumannetal2009:naacl} and many of the current freely-accessible \asr\ systems can produce output (semi-) incrementally. 

In our \sds, we opt for Google \asr\ because of its wide vocbaulary coverage of the language we are interested in (German). We are able to package \asr\ output from the Google service into \iu s as explained above. Those word \iu s are passed to the \nlu\ module, which will now be explained. 

\subsection{Language Understanding}

We approach the task of \nlu\ as a slot-filling task (a very common approach; see \cite{Tur2012}) where the system can fill the task when all slots of a frame are filled. The main driver of the \nlu\ in our \sds\ is the \sium\ model of \nlu\ introduced in \cite{Kennington2013a}. \sium\ has been used in several systems which have reported impressive results in various domains, languages, and tasks \cite{Kennington2014_coling,Kennington2015_naacl}. Though originally a model of reference resolution, the authors hinted that it could be used for general \nlu, which we do here. The model is formalised as follows:

\begin{center}
\begin{equation}
   P(I|U) = \frac{1}{P(U)} P(I)\sum_{r\in R} P(U|R)P(R|I) 
\label{eq:disc1}
\end{equation}
\end{center}

That is, $P(I|U)$ is the probability of the intent $I$ (i.e., a frame slot) behind the speaker's (ongoing) utterance $U$. This is recovered using the mediating variable $R$, a set of \emph{properties} which map between aspects of $U$ and aspects of $I$. These properties could be visual properties of visible objects, or they could be more abstract properties that intents might have, which we opt for here (e.g., the intent of a \texttt{restaurant} might be filled by a certiain type of cusine such as \texttt{italian} which has (among others) properties like \texttt{pasta}, \texttt{mediteranian}, \texttt{vegetarian}, etc.). Properties are pre-defined by a system designer and can match words that might be uttered to describe the intent in question. The mapping betwen properties and aspects of $U$ can be learned from data. During application, $P(U|R)$ can produce a distribution over words (or properties, see below) which are summed over and the probability mass for each property is accumulated for each intent, resulting in a distribution over possible intents. This occurs at each word increment, where the distribution from the previous increment is combined via $P(I)$, keeping track of the distribution over time.  

In our \sds, we apply an instantion of \sium\ for each slot (explained in Section~\ref{section:experiments}), all of which update at each word increment. At each word increment, the updated slots (and their corresponding) distributions are given to the \dm, which will now be explained. 

\subsection{Dialogue Manager}

The primary job of our \dm\ is to determine \emph{when} to act, given the unfolding utterance. That is, at each word, the \dm\ needs to choose one of the following:
\begin{itemize}
 \item \texttt{wait} -- don't act until more information is forthcoming
 \item \texttt{select} -- the \nlu\ is confident enough that a slot can be filled
 \item \texttt{request} -- the dialogue has reached a state where the system has asked for a (yes/no) clarification request
 \item \texttt{confirm} -- the user has responded to the clarification request (which acts similar to a \texttt{select})
\end{itemize}

This is a crucial part to our \sds\ which sets it apart from other systems in that the \dm\ is called upon at each word to decide \emph{when} to act, rather than \emph{how} to act, effecively giving the \dm\ the control over timing of actions rather than relying on \asr\ endpointing. The \dm\ policy is based on a confidence score (\conf) derived from the \nlu\ (in this case, we used the distribution's argmax value) suing thresholds for \texttt{wait}, \texttt{confirm} and \texttt{select} set by hand (i.e., trial and error). Though the thresholds were statically set, we applied OpenDial \cite{Lison2015a} as a modularised iu-module to perform the task of the \dm\ with the future goal that these values could be adjusted through reinforcement learning (which OpenDial could provide). The \dm\ processes and makes a decision for \emph{each slot}, with the assumption that only one slot out of all that are processed will result in an non-\texttt{wait} action (though this is not constrained). The candidate slots that are processed depends on the state of the \ui\ (described below); only slots represented by visible nodes are considered, thereby reducing the possible frames that could be predicted.

\subsection{Graphical Interface}
\label{section:display}

The goal of the display is to inform the user about the internal state of the ongoing understanding. One motivation for this is that the user can determine if the system understood the user's intent before providing the user with a response (e.g., a list of restaurants of a certain type)--i.e., if any misunderstanding takes place, it happens before the system commits to an action and is potentially more easily repaired. 

\begin{wrapfigure}{r}{0.25\textwidth}
  \centering
      \includegraphics[width=0.25\textwidth]{figures/diatree-affordances.png}	
      \caption{Example tree as branching from the root; each branch represents a system affordance (i.e., making a phone call, reminder, finding a restaurant, leaving a message, and finding a route). \label{fig:afforances}}
\end{wrapfigure}

The display is a right-branching tree, where the branches directly off the root node display the afforances of the system (i.e., what domains of things it can understand and do something about). When the first tree is displayed, it represents a state of the \nlu\ where none of the slots are filled, an example of which is shown in Figure~\ref{fig:afforances}. 

When a user selects a domain to ask about, the tree is adjusted to make that domain the only one displayed and the slots that are required for that domain are shown as branches. The user can then fill those slots (i.e., branches) by uttering the name of the slot, or, alternatively, by uttering the item to fill slot directly. For example, at a minimum, the user could utter the name of the domain then an item for each slot (e.g.,  \emph{food French downtown}) or the speech could be more natural (e.g., \emph{I'm quite hungry, I am looking for some French food maybe in the downtown area}). When something is uttered that falls into th \texttt{confirm} state of the \dm\ as explained above, the display expands the subtree under question (\todo{bring QUD into this?}) and marks the item with a question mark. And example of this is shown in Figure~\ref{fig:confirm} where \emph{franzözisch} (French) is requested to be confirmed. At this point, the user can utter any kind of confirmation. A positive confirmation would fill the slot with the item in question, collapsing that particular branch of the tree. A negative confirmation would retract the question, but leave the branch expanded. The expanded branches are displayed according to their rank as given by the \nlu's probability distribution.

\begin{figure}[ht]
  \centering
      \includegraphics[width=0.5\textwidth]{figures/diatree-confirmation.png}	
      \caption{Example tree asking for confirmation on a specific node (in red with a question mark).\label{fig:confirm}}
\end{figure}

A filled  branch is collapsed, visually marking it as filled. At any time, a user can backtrack by saying \emph{no} (or equivalent) or start the entire interaction over from the beginning with a keyword, e.g., \emph{restart}. To aid the user's attention, the node under question is marked in red, where filled slots are represnted by blue nodes, and filled nodes represent candidates for the current slot in question. For cases where the system is in the \texttt{wait} state for several words (during which there is no change in the tree), the system signals activity at each word by causing the red node in question to temporarily change to white, then back to red (i.e., appearing as a blinking node to the user). Figure~\ref{fig:filled} represents a filled frame represnted as tree with one branch for each filled slot.

\begin{figure}[ht]
  \centering
      \includegraphics[width=0.5\textwidth]{figures/diatree-filled.png}	
      \caption{Example tree where all of the slots are filled. (i.e., \texttt{domain:food}, \texttt{location:nearby}, \texttt{type:french}) \label{fig:filled}}
\end{figure}

Such an interface clearly shows the internal state of the \sds; whether or not it has understood the request so far. It is desinged to aid the user's attention to the slot in question, and clearly indicates the affordances that the system has. At the moment the interface is simply a read-only display that is purely speech-driven, but it could be augmented with additional funcionalities, such as tapping a node for expansion or typing intput that the system might not yet display. It is currently implemented as a web-based interface (using the javascript D3 library), allowing it to be usable as a web application on any machine or mobile device. 

\section{Experiments}
\label{section:experiments}

In this section, we describe two experiments in which we evaluated our system. In order to best evaluate our system, we recruited participants to interact with our system in varied settings to compare non-increemntal and incremental as well as adaptive versions. We will describe how the data were collected from the participants, then explain each experiment and give results.

\subsection{Task \& Instructions} 
The participants were seated at a desk and given written instructions that they were to use the system to perform as many tasks as possible in the allotted time. The instructions gave them several examples of kinds of tasks the system could understand with an icon representing the domain and then text for the rest of the task, as shown in Figure~\ref{fig:taskex}. In front of the participant towards the rear of the table was a computer screen that would show the task in the middle of the screen. In front of that screen closer to the participant was a small tablet that showed the \ui.\footnote{We used a Samsung 8.4 Pro turned to its side to show a larger width for the tree to grow to the right.} The user was instructed to convey the task presented on the screen to the system such that the \ui\ on the tablet would have a completed tree (i.e., the tree was in its filled state as in Figure~\ref{fig:filled}). When the participant was satisfied that the system understood her intent, she was to press spacebar on a keyboard in front of the tablet which triggered a new task to be displayed and the screen and reset the tree to its start state on the tablet (which was the root node linking to the 5 possible tasks). See Figure~\ref{fig:dataview} for an overview of the experiment setup.

\begin{wrapfigure}{r}{0.3\textwidth}
  \centering
      \includegraphics[width=0.3\textwidth]{figures/taskexample.png}	
      \caption{Examples of tasks, as presented to each participant. Each icon represents a specific task domain (e.g., \emph{route} finding).\label{fig:taskex}}
\end{wrapfigure}

The possible tasks were \emph{call}, which had a single slot for \emph{name} to be filled (i.e., one out of the 22 most common German given names); \emph{message} which had a slot for \emph{name} and a slot for the \emph{message} (which, when invoked, would simply fill in directly from the \asr\ until 1 second of silence was detected); \emph{eat} which had slots for \emph{type} (in this case, 6 possible types) and \emph{location} (in this case, 6 possible known locations based around the city of Bielefeld); \emph{route} which had slots for \emph{source} city and the \emph{destination} city; and \emph{reminder} which had a slot for \emph{message}. Some of the slot types were shared across domains, such as \emph{message} (across \emph{reminder} and \emph{message}), \emph{name} (across \emph{message} and \emph{call}), as well as \emph{source} and \emph{destination} which shared the same list of German city names (the top 100 most populous cities). 

The tasks presented to the participant were randomly chosen: first, the domain was randomly chosen from the 5 possible domains, and then each slot value to be filled was randomly chosen (the \emph{message} slot for the \emph{name} and \emph{messag} domains was randomly selected from a list of 6 possible ``messages'', each with 2-3 words; e.g., \emph{feed the cat}, \emph{visit grandma}, etc.). The system kept track of which tasks were already presented to the participant. At any time after the first task, the system could choose a task that was previously presented and presented it again to the participant (with a 50\% chance) so the user would often see tasks that she had seen before (with the assumption that humans who use \pa s often do perform similar, if not the same, tasks more than once). 

\begin{figure}[ht]
  \centering
      \includegraphics[width=0.5\textwidth]{figures/dataview.pdf}	
      \caption{Bird's eye view of the experiment: the participant sat at a table with a screen, tablet, and keboard in front of them. \label{fig:dataview}}
\end{figure}

The participant was told that she would interact with the system in three different phases, each for 4 minutes, and was instructed to accomplish as many tasks as possible in that time allotment. The participant was not told what the diffrent phases were, only that the system was somewhat different in each phase. The experiments described in Sections~\ref{section:exp1} and \ref{section:exp2} respectively describe and report a comparison first between the first and second phase (denoted the non-incremental and incremental variants of the system) and a comparison between the incremental and incremental-adaptive phases. Each of these phases are described below. Before the participant began Phase 1, they were able to try it out for up to 4 minutes (in Phase 1 settings) and ask questions about how it worked, allowing them to get used to the interface before the actual experiment began. After this trial phase, the experiment began.

\paragraph{Phase 1: Non-incremental} In this phase, the system did not appear to work incrementally; that is, the system only displayed tree updates after \asr\ endpointing (of 1.2 seconds, which is a resonable amount of time to expect a response from a commercial spoken \pa). The system displayed the ongoing \asr\ on the tablet as it was recognised (as is often done in commercial \pa s). The participant knew that the system fully understood the task when the displayed tree had no more unfilled branches (as in Figure~\ref{fig:filled}). 

\paragraph{Phase 2: Incremental} In this phase, the system displayed the tree information incrementally as explained above. The \asr\ was no longer displayed; only the tree provided feedback in understanding. 

After Phase 2, a 10-question questionnaire was displayed on the screen for the participant to fill out comparing Phase 1 and Phase 2. For each question, they had the choice of \emph{Phase 1}, \emph{Phase 2}, \emph{Both}, and \emph{Neither}. (See appendix for full list of questions.) 

\paragraph{Phase 3: Incremental-adaptive} In this phase, the incremental system was again presented to the participant with an added user model that ``learned'' about the user. If the user saw a task more than twice, the user model would predict that, if the user chose that task domain again (e.g., \emph{route}) then the system would automatically ask a clarification on the slots. If the user saw a task more than 3 times, the system skipped asking for clarifications and filled in the domain slots completely, requiring the user only to press the spacebar to confirm it was the correct one (i.e., to end the task). An example progression might be as follows: a participant is presented with the task \emph{route from Bielefeld to Berlin}, then the user would attempt to get the system to fill in the tree (i.e., slots) with those values. After some interaction in other domains, the user sees the same task again. After additional interaction in other domains, the task is repeated and here the user must only say ``yes'' for each slot to confirm the system's prediction. Later, if the task is presented a fourth time, the user would enter that domain (i.e, \emph{route}) and the two slots would already be filled. If later a new route task was presented (e.g., \emph{route from Bielefeld to Hamburg} the system would already have the slots filled for \texttt{from:Bielefeld}, \texttt{destination:Berlin}, but the user could backtrack by saying ``no, to Hamburg'' which would trigger the system to fill \texttt{destination:Hamburg}. Later interactions within the \emph{route} domain would ask for a clarification on the \emph{destination} slot since it has had several possible values given by the participant. 

After Phase 3, the participants were presented with another questionnaire on the screen to fill out with the same questions (plus two additional questions), this time comparing Phase 2 and Phase 3. For each question, they had the choice of \emph{Phase 2}, \emph{Phase 3}, \emph{Both}, and \emph{Neither}.

At the end of the three phases and questoinnaries, the participants were given a final questionnaire to fill out by hand on their general impressions of the systems. 

We recruited 14 participants to participate in the evaluation. We used the Mint tools data collection framework described in \cite{kousidis2012evaluating} to log the interactions. Due to some technical issues, one of the participants did not log interactions, we collected data from 13 participants, post-Phase 2 questionnaires from 12 participants,  post-Phase 3 questionnaires from all 14 participants, and general questoinnaires from all 14 participants. In the experiments that follow, we report object and subjective measures to determine the settings that produced superior results.


\subsection{Experiment 1: Non-Incremental vs. Incremental}
\label{section:exp1}

\subsection{Results}

\subsection{Experiment 2: Incremental vs. Incremental-Adaptive}
\label{section:exp2}

\subsection{Results}

\section{Conclusion \& Future Work}

\section*{Appendix}

The following questions were asked on both questoinnaries following Phase 2 and Phase 3 (comparing the two most latest used system versions):
\begin{itemize}
 \item The interface useful and easy to understand.
 \item The assistant was easy and intuitive to use.
 \item The assistant understaood what I wanted to say.
 \item I always understood what the system wanted from me. 
 \item The assistant made many mistakes. 
 \item The assistant did not respond while I spoke.
 \item It was sometimes unclear to me if the assistant understood me. 
 \item The assistant responde while I spoke. 
 \item The assistant sometimes did things that I did not expect.
 \item When the assistant made mistakes, it was easy for me to correct them. 
\end{itemize}

In addition to the above 10 questions, the following were also asked on the questionnaire following Phase 3:
\begin{itemize}
 \item I had the feeling that the assistant attempted to learn about me.
 \item I had the feeling that the asssitant made incorrect guesses. 
\end{itemize}


The following questions were used on the general questoinnaire (as translated into English):
\begin{itemize}
 \item I regularly use peronal assistants such as Siri, Corana, Google now or Amazon Echo: Yes/No
 \item I have never used a speech-based personal assistant: Yes/No
 \item What was your general impression of our personal assistants?
 \item Would you use one of these assistants on a smartphone or tablet if it were available? If yes, which one?
 \item Do you have suggestions that you think would help us improve our assistants?
 \item If you have used other speech-based interfaces before, do you prefer this interface?
\end{itemize}


\bibliographystyle{acl2016}
\bibliography{refs}

\end{document}

